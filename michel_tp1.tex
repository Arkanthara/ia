\documentclass{article}

\title{Série 1: Formalisation}
\author{Michel Donnet}
\date{\today}

\begin{document}
\section{Les trois médecins}

\subsection{Description}

Alice, Bob et Charles sont trois patients suivis par les médecins Xavier,
Yolande et Zoé. Aujourd’hui, ces 6 personnes doivent être transférées de
l’Hôpital Guéritou vers l’Hôpital Soignetou grâce à une ambulance proposant
deux places. Les médecins ne doivent jamais se retrouver en infériorité
numérique pour prodiguer leur soins. On peut toutefois avoir des patients
dans un hôpital sans médecins. De plus l’ambulance ne fait pas de trajet à
vide.
Comment va-t-il falloir organiser les diférents trajets pour transporter
les 6 personnes d’un hôpital à l’autre, en respectant les contraintes ?

\subsection{Formalisation du problème recherche}
Formalisez le problème en extrayant l’information suisante pour représenter
les diférentes situations possibles. Proitez de cet exercice pour reviser les
notions du cours.

\begin{itemize}
  \item 1 Donnez une représentation des états.
  \item 2 Quels sont les opérateurs de transition possibles ?
  \item 3 Définissez les conditions pour lesquelles les opérateurs sont applicables.
  \item 4 En vous basant sur le TP0, implementez un algorithme de recherche
pour résoudre le problème en utilisant un arbre de recherche correspondant à la description que vous avez choisi.
\end{itemize}


\end{document}
